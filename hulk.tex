% easychair.tex,v 3.5 2017/03/15

\documentclass{easychair}
%\documentclass[EPiC]{easychair}
%\documentclass[EPiCempty]{easychair}
%\documentclass[debug]{easychair}
%\documentclass[verbose]{easychair}
%\documentclass[notimes]{easychair}
%\documentclass[withtimes]{easychair}
%\documentclass[a4paper]{easychair}
%\documentclass[letterpaper]{easychair}

\usepackage{doc}
\usepackage{fullpage}

% use this if you have a long article and want to create an index
% \usepackage{makeidx}

% In order to save space or manage large tables or figures in a
% landcape-like text, you can use the rotating and pdflscape
% packages. Uncomment the desired from the below.
%
% \usepackage{rotating}
% \usepackage{pdflscape}

% Some of our commands for this guide.
%
\newcommand{\easychair}{\textsf{easychair}}
\newcommand{\miktex}{MiK{\TeX}}
\newcommand{\texniccenter}{{\TeX}nicCenter}
\newcommand{\makefile}{\texttt{Makefile}}
\newcommand{\latexeditor}{LEd}

%\makeindex

%% Front Matter
%%
% Regular title as in the article class.
%
\title{Porting the Mathematical Components library to Hierarchy Builder}

% Authors are joined by \and. Their affiliations are given by \inst, which indexes
% into the list defined using \institute
%
\author{
  Reynald Affeldt\inst{3}
  \and
  Xavier Allamigeon\inst{4}
  \and
  Yves Bertot\inst{1}
  \and
  Quentin Canu
  \and
  Cyril Cohen\inst{1}
  \and
  Marie Kerjean
  \and
  Pierre Roux
  \and
  Kazuhiko Sakaguchi\inst{2}
  \and
  Enrico Tassi\inst{1}
  \and
  Laurent Thery\inst{1}
  \and
  Anton Trunov
}

% Institutes for affiliations are also joined by \and,
\institute{
  Inria, Universit\'e c\^ote d'Azur
\and
   Tsukuba University
\and
   AIST
\and
   Inria, CMAP, CNRS, Ecole Polytechnique, Institut Polytechnique de Paris
 }

%  \authorrunning{} has to be set for the shorter version of the authors' names;
% otherwise a warning will be rendered in the running heads. When processed by
% EasyChair, this command is mandatory: a document without \authorrunning
% will be rejected by EasyChair

\authorrunning{The Mathematical Components developers}

% \titlerunning{} has to be set to either the main title or its shorter
% version for the running heads. When processed by
% EasyChair, this command is mandatory: a document without \titlerunning
% will be rejected by EasyChair
\titlerunning{Mathematical Components and Hierarchy Builder}

\begin{document}

\maketitle

% The table of contents below is added for your convenience. Please do not use
% the table of contents if you are preparing your paper for publication in the
% EPiC Series or Kalpa Publications series

%\setcounter{tocdepth}{2}
%{\small
%\tableofcontents}

%\section{To mention}
%
%Processing in EasyChair - number of pages.
%
%Examples of how EasyChair processes papers. Caveats (replacement of EC
%class, errors).

%------------------------------------------------------------------------------
\section{Context}
\label{sect:introduction}

This abstract reports on the porting of the Mathematical Components library
to the Hierarchy Builder~\cite{cohen_et_al:LIPIcs:2020:12356} tool.

Mathematical Components is an extensive and coherent repository of formalized
mathematical theories. At the time of writing about 40 opam packages depend
on some components from this library and these packages are not exclusively
focusing on mathematics (eg Iris, QuickChick, Disel \ldots). 
The library is made of 91 files for a total 124 thousands lines of code.

The key to keep the library growing in a rational way is that it revolves
around a hierarchy of interfaces which organizes operations and properties.
Interfaces come with theories which apply, automatically, to all the objects
which are registered as validating the interface. These interfaces are
implemented following the packed classes discipline which works well in practice
but has never been easy to master. Pull Requests extending or modifying the
hierarchy turned out to be problematic, hard to review and integrate. Only
a few were merged, the others are put on hold waiting for the library begin
ported to Hierarchy Builder.

Hierarchy Builder is a set of commands implemented in Coq-Elpi which take
care of automatically synthesizing all the error prone boilerplate required for
packed classed to work. The user just inputs record declarations,
standing for the interfaces, and proofs, for the instances. She is relieved from
most of the gibberish of modules, sections, coercions, canonical
structures, implicit arguments, phantom abbreviations, \ldots
 
In the week from the 6th to the 9th of April 2021 the authors of this abstract
gathered (virtually) for a sprint on porting the Mathematical Components to
Hierarchy Builder. The whole library finally compiled on April 20.

The porting consisted in declaring all the interfaces and their instances
using the new command, \emph{without changing proofs} in any substantial way.
At the time of writing the patch~\footnote{\url{https://github.com/math-comp/math-comp/pull/733}}
amounts to about 5K lines being edited, and 5K lines being simply \emph{removed}.

\section{Report}

\begin{itemize}
\item performance of term comparison and abstraction barriers, locking
\item not all design patterns are supported, new ones invented
\item not all structures can be dealt with, a part of ssrnum is currently half disabled
\item documentation experiments (graph, docstring, generation)
\end{itemize}

\section{Plan for Mathematical Components version 2.0}

\begin{itemize}
\item 1.x releases will continue for a while and will
\item 2.x when ready, 1.x still there in maintenance mode
\item dev focus on 2.x, 1.x for legacy
\end{itemize}

\label{sect:bib}
\bibliographystyle{plain}
%\bibliographystyle{alpha}
%\bibliographystyle{unsrt}
%\bibliographystyle{abbrv}
\bibliography{bib}

%------------------------------------------------------------------------------
\end{document}

